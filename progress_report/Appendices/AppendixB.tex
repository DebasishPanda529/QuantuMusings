\chapter{Bloch's Theorem}

Periodic potentials are important in condensed matter physics, and we will be using the Bloch wavefunctions generously during the analysis of toy models. Secondly, periodic potentials will give us our first examples of Hamiltonian systems with symmetry, and they will serve to illustrate certain general principles of such systems. \\

We wish to solve the one-dimensional Schrödinger equation,

\begin{equation*}
    -\frac{\hbar^{2}}{2m}\psi^{\prime \prime} + V(x) \psi = E \psi
\end{equation*}

where the potential is assumed to be spatially periodic,

\begin{equation}
    V(x+a) = V(x)
\end{equation}

Here $a$ is the lattice spacing or spatial period of the 1-D lattice. No further assumptions need be made about the behaviour of $V(x)$ within any period apart from its periodicity. \\

Next, we shall make a strong assumption that there is a super-symmetry that rides over the good ole periodicity of the lattice points such that the lattice repeats itself after $N$ lattice spacings. This is equivalent to imposing a periodic/circular boundary condition on the solutions to the Hamiltonian. \\

We introduce the translation operator, $T(a)$, which has the effect of displacing the wave function by the lattice spacing $a$ along the x-axis. 

\begin{equation}
    T(a) \psi(x) = \psi (x-a)
\end{equation}

Functionally, the translation operator is given by,

\begin{equation}
    T(a) = e^{-\frac{iap}{\hbar}}
\end{equation}

An easy check will ascertain that this operator commutes with both kinetic energy, as well as potential energy operators. This means that \textit{T(a)} commutes with the entire Hamiltonian, 

\begin{equation}
    [T(a), H] = 0
\end{equation}

Put more generally, $H$ commutes with any power of $T(a)$, $T(a)^{n} = T(na)$, which is to say that it commutes with the entire group of symmetry operations generated by $T(a)$. \\

The fact that $H$ and $T(a)$ commute provides us a powerful tool to determine the eigenfunctions of $H$. More often than not, it is hard to find the eigenfunctions of $H$, but much easier to find those for the translation operator. Since we now know the eigenfunctions of the translation operator, it makes the search for the eigenfunctions of $H$ easier since they are a subset of the eigenspace of $T(a)$. \\

Since $T(a)$ is unitary, its eigenvalue $\tau$ must be a phase factor, $\tau = e^{-i\theta}$. The angle $\theta$ characterizes the eigenvalues of $T(a)$ and may be restricted to the range $-\pi < \theta \leq \pi$. It is conventional to write this angle in the form $\theta = ka$, where $k$ is a quantity with dimensions of wave number,
which characterizes the eigenvalue. We now have,

\begin{equation}
    T(a) \psi_{k}(x) = \psi_{k}(x-a) = e^{-ika}\psi_{k}(x)
\end{equation}

Equivalently, we can write this as,

\begin{equation}
\label{eq:bloch wavefunction}
    \psi_{k}(x+a) = e^{ika}\psi_{k}(x)
\end{equation}

Now we are faced with a dilemma - for any given value of $k$, there are functions $\psi_{k}$ which satisfy \ref{eq:bloch wavefunction}, so the spectrum of $T(a)$ is the entire unit circle in the complex plane. Furthermore, the number of such functions for any value of $e^{-ika}$ is infinite, so the eigenvalues are infinite-fold degenerate and the eigenspaces of $T(a)$ are infinite-dimensional. This would render the entire analysis using translation operators inconsequential since it was asserted that this approach would help limit the space in which we have to search for the eigenfunctions of $H$. This is exactly where the initial boundary condition assuming a super-symmetry comes into play. In case the lattice repeats itself after $N$ lattice spacings, the single-valuedness of the wavefunction requires

\begin{equation*}
    \psi(x+Na) = \psi(x)
\end{equation*}

so the eigenvalues of $T(a)$ are phase factors of the form $e^{-\frac{2n\pi i}{N}}$, for $n = 0,..., N-1$. In this case, the spectrum of $T(a)$ is discrete, although each eigenvalue is still infinite-fold degenerate. Rather than $\psi_{k}(x)$, it is often easier to work with a function $u_{k}(x)$, defined by

\begin{equation}
    \psi_{k}(x) = e^{ikx} u_{k}(x)
\end{equation}

where $u_{k}$ is periodic, $u_{k}(x+a) = u_{k}(x)$. \textit{Bloch's theorem} states that since $H$ commutes with $T(a)$, $H$ possesses eigenfunctions which are of the form of $\psi_{k}(x)$, that is, $e^{ikx}$ times a periodic function $u_{k}(x)$. \\

An interesting offshoot of the Bloch wavefunction is the concept of 'crystal momentum', which does not represent the momentum of the electron in real space but rather encapsulates the effect of the net external potential acting on it without having to concern ourselves with the internal forces. 