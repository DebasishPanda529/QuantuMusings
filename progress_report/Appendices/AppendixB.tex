\chapter{Non-Equilibrium Green's Function Formalism (NEGF)}

This appendix presents an algorithmic approach employed for simulations using the NEGF framework. \par

The retarded Green's function in the energy domain is given by

\begin{equation}
	G^{r}(E) = [(E+i\eta)\mathds{1}-\mathcal{H}-\Sigma_{L}-\Sigma_{R}]^{-1}
\end{equation}

where $\mathcal{H}$ is the contact Hamiltonian, $\Sigma_{L,R}$ are the self-energies of the semi-infinite contacts, and $\eta$ is an infinitesimally small damping parameter. The advanced Green's function is defined as the Hermitian conjugate of the retarded Green's function ($G^{a} = G^{r\dagger}$). The surface Green's functions ($g_{s}$) at each contact are recursively evaluated 

\begin{equation}
	g_{si}(E) = [(E+i\eta)\mathds{1}-\alpha_{i}-\beta^{\dagger}_{i}g_{si}(E)\beta_{i}]^{-1}
\end{equation}

where the subscript $i$ labels the contact. The lead self-energy matrices $\Sigma_{L}$ and $\Sigma_{R}$ are computed as 

\begin{equation}
    \Sigma_{1} = 
    \begin{pmatrix} 
        \sigma_1 & 0 \\ 
        0 & \mathbf{0}
    \end{pmatrix} , \:
    \Sigma_{2} = 
    \begin{pmatrix}
        \mathbf{0} & 0 \\ 
        0 & \sigma_{2}
    \end{pmatrix}
\end{equation}

where $\sigma_{i} = \beta g_{si}\beta^{\dagger}$ \par

The anti-Hermitian part of the self-energy is responsible for the finite life-time of the quasiparticles in the junction and broadens the energy levels. This broadening matrix is denoted by $\Gamma_{i}$. \par

The Fermi functions in the particle-hole Nambu space is given by

\begin{equation}
	F_{i} = 
	\begin{bmatrix}
	 	f(E,\mu+eV) & 0 \\
	 	0 & f(E,-\mu-eV)
	 \end{bmatrix} 
\end{equation}

where $f(E,\mu)$ is the Fermi function, and $V$ is the contact bias. \par

The lesser self-energy, or the inscattering matrix can be computed from the broadening matrix and Fermi functions as 

\begin{equation}
	-i\Sigma^{<} = \Sigma^{in} = \Gamma_{1}F_{1} + \Gamma_{2}F_{2}
\end{equation}

The lesser Green's function is computed as

\begin{equation}
	-iG^{<} = G^{n} = G^{r}\Sigma_{in}G^{a}
\end{equation}

Transport properties of the setup are calculated from the elements of the current operator matrix. The current operator through the left cont

\begin{equation} \label{eq:current}
	I^{op}_{L}(E) = \frac{ie}{h}\:(G^{r}\Sigma_{L}^{in}-\Sigma_{L}^{in}G^{a}-\Sigma_{L}G^{n}+G^{n}\Sigma_{L}^{\dagger})
\end{equation}

The current operator in \ref{eq:current} signifies electron currents and taking its trace gives us the well-known current operator formula which reads

\begin{equation}
	I^{op}_{L}(E) = \frac{e}{h}\:\text{\bf{Tr}}[\Sigma^{in}_{L}A-\Gamma_{L}G^{n}]
\end{equation}

where $A(E) = i(G^{r}(E)-G^{a}(E))$, is the spectral function. \par

The current operator must be suitably modified in Nambu space and all quantities must be consistent with the BdG Hamiltonian. The net current is given by the difference of the partial trace of the current operator over the electron and hole sub-spaces.

\begin{equation}
   \begin{aligned}
      J(E) &= \text{\bf{Tr}}_{e}(I_{op})-\text{\bf{Tr}}_{h}(I_{op})	\\
      &= \text{\bf{Tr}}(I_{op}\tau_{z})
   \end{aligned}
\end{equation} 

where $\tau_{z} = \sigma_{z}\otimes  \mathds{1}_{N\times N}$ is the Pauli-$z$ operator in the particle-hole Nambu space. The total current is then evaluated by integrating the current-energy density

\begin{equation}
	I(\phi) = \int_{-\infty}^{\infty}J(E) dE
\end{equation}