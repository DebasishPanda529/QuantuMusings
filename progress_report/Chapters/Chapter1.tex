\chapter{Second Quantisation}

Before we begin our incursion into the so-called 'second quantisation', we need to appreciate the reason why the need for second quantization arose. The properties of quantum condensed matter systems and, by extension, that of real materials are controlled by the \textit{collective behaviour} of electrons in the presence of some background potential due to an underlying crystal lattice. This statement, in fact, is a simpler rendition of the Bohr-Oppenheimer approximation. So, what factors do we need to consider during the analysis of a condensed matter system?
\begin{itemize}
    \item Focus on electrons and their collective dynamics
    \item Electrons are free to move from one orbital to another (tunnelling/hopping)
    \item They are subject to a background potential from the lattice
    \item They can interact with each other due to Coulomb repulsion
\end{itemize}
The question remains, how do we formulate the Hamiltonian for many-body systems? How do we encode anti-symmetry of fermions into this many-particle wavefunction? And most importantly, how do we find out the eigenstates/eigenvalues of momentum and/or energy of the system?  \par

So, how do we encode fermionic anti-symmetry in many-particle wavefunctions? \\
Consider a single-particle quantum state $\phi_{\nu}(\vec{r})$, where $\nu$ refers to labels for the quantum state. The basis for a two-particle system is then given by 

\begin{equation*}
    \psi(\vec{r_{1}}, \vec{r_{2}}) = \frac{1}{\sqrt{2}}[\phi_{\nu_{1}}(\vec{r_{1}}) \phi_{\nu_{2}}(\vec{r_{2}}) - \phi_{\nu_{1}}(\vec{r_{2}}) \phi_{\nu_{2}}(\vec{r_{1}})]
\end{equation*}

This basis satisfies the anti-symmetry property, and also, there happens to be a less verbose manner through which we can express such wavefunctions - Slater's determinants. \\
For a generalized $N$-particle system such that the basis states are perfectly anti-symmetric under exchanging the labels of any two particles, the wavefunction can be expressed as

\begin{equation*}
    \psi(\vec{r_{1}}, \vec{r_{2}},..., \vec{r_{N}}) = \frac{1}{\sqrt{N!}} \begin{bmatrix}
        \phi_{\nu_{1}}(\vec{r_{1}}) & \phi_{\nu_{2}}(\vec{r_{1}}) & ... & \phi_{\nu_{N}}(\vec{r_{1}}) \\
        \phi_{\nu_{1}}(\vec{r_{2}}) & \phi_{\nu_{2}}(\vec{r_{2}}) & ... & \phi_{\nu_{N}}(\vec{r_{2}}) \\
        . & . &  & . \\
        . & . &  & . \\
        . & . &  & . \\
        \phi_{\nu_{1}}(\vec{r_{N}}) & \phi_{\nu_{2}}(\vec{r_{N}}) & ... & \phi_{\nu_{N}}(\vec{r_{N}})
    \end{bmatrix}
\end{equation*}

The 'first quantisation' principle cannot be used to satisfactorily explain condensed matter systems since calculations become cumbersome and expensive as the number of particles in the system increases, and the representation requires the number of particles, $N$, to be fixed. As $N$ approaches the limit associated with statistical physics, $N$ is allowed to fluctuate as per the grand canonical ensemble. Second quantisation or occupation number formalism is the standard way in which many-particle QM is formulated. It is based on the algebra of ladder operators.

\begin{itemize}
    \item Second quantisation provides a compact way of representing the many-body space of excitations.
    \item Properties of operators are now encoded in a single set of commutation/anti-commutation relations rather than in some explicit Hilbert space representation. 
\end{itemize}

In essence, second quantisation formalism offers us significant computational advantage and a more compact and efficient representation of the Hamiltonian when dealing with many-particle quantum systems. For example, consider a \textit{symmetrised} $N$-particle wwavefunction of fermions ($\zeta = -1$) or bosons ($\zeta = +1$) expressed in the form 

\begin{equation}
	\ket{\lambda_{1}, \lambda_{2},\ldots \lambda_{N}} = \frac{1}{\sqrt{N! \prod_{\lambda=0}^{\infty}n_{\lambda}!}} \sum_{\mathcal{P}}\zeta^{\mathcal{P}} \ket{\psi_{\lambda_{\mathcal{P}1}}} \otimes \ket{\psi_{\lambda_{\mathcal{P}2}}} \ldots \otimes \ket{\psi_{\lambda_{\mathcal{P}N}}}
\end{equation}

where $n_{\lambda}$ is the total number of particles in state $\lambda$ (for fermions, Pauli exclusion principle dictates that $n_{\lambda} = 0,1$, i.e. $n_{\lambda}! = 1$). The summation runs over all $N!$ permutations of the quantum numbers $\lambda_{i}$, and $\mathcal{P}$ denotes the parity. \\ \footnote{Parity is defined as the number of transpositions of two elements which brings the permutation ($\mathcal{P}_1, \mathcal{P}_2,\ldots \mathcal{P}_{N}$)) back to the ordered sequaence (1,2,\ldots N)}

Second quantisation formalism provides for a much more condensed and intuitive representation for the generalised wavefunction via the \textbf{vacuum state} $\ket{\Omega}$, and a set of creation (annihilation) \textbf{field operators} $c_{\lambda}$ ($c_{\lambda}^{\dagger}$), as follows:

\begin{equation}
\label{eq:eq1}
    c_{\lambda} \ket{\Omega} = 0, \quad \frac{1}{\sqrt{\prod_{\lambda}n_{\lambda}!}} c_{\lambda_N}^{\dagger} \ldots c_{\lambda_1}^{\dagger} \ket{\Omega} = \ket{\lambda_{1}, \lambda_{2},\ldots \lambda_{N}}
\end{equation}

In terms of physical interpretation, the operator $c_{\lambda}^{\dagger}$ creates a particle in state $\lambda$ while the operator $c_{\lambda}$ annihilates it. The commutation relations between these operators are captured via Clifford algebra \footnote{$[\hat{A},\hat{B}]_{-\zeta} = \hat{A}\hat{B}-\zeta \hat{B}\hat{A}$ is the commutator $\zeta = 1$ (anticommutator $\zeta = -1$) for bosons (fermions). As per convention, the notation [.,.] denotes the commutator while \{.,.\} the anticommutator.}:

\begin{equation}
    [c_{\lambda},c_{\mu}^{\dagger}]_{-\zeta} = \delta_{\lambda,\mu}, \quad [c_{\lambda},c_{\mu}]_{-\zeta} = [c_{\lambda}^{\dagger},c_{\mu}^{\dagger}]_{-\zeta} = 0
\end{equation}

The physical interpretation of \ref{eq:eq1} and the commutation relations of the field operators is no trivial matter -- these equations imply that for \textit{any N}, the $N$-body wavefunction can be generated by an application of a set of $N$-independent operators to a vacuum state. Similarly, the formal definition of the general many-body or \textbf{Fock space} can be given as the direct sum $\oplus_{N=0}^{\infty}\mathcal{F}_N$, where $\mathcal{F}_N$ is defined as the linear span of all $N$-particle states $\ket{\lambda_{1}, \lambda_{2},\ldots \lambda_{N}} = \frac{1}{\sqrt{\prod_{\lambda}n_{\lambda}!}} c_{\lambda_N}^{\dagger} \ldots c_{\lambda_1}^{\dagger} \ket{\Omega}$. Intuitively, the Fock-subspaces $\mathcal{F}_N$ are generated by repeated action of creation operators on the vacuum space $\mathcal{F}_0$, and application of creation/annihilation field operator on a wavefunction takes it from one Fock-subspace to another. \\ 

Before proceeding further, we need to determine the basis transformation for the field operators, and the Fourier transform of the operators from the real space to $k$-space (otherwise known as the momentum space). These results will prove incredibly useful while analysing the Hamiltonians for interacting fermionic systems. \\

\clearpage

\subsection{Change of basis}

The identity operator, $\mathcal{I}$ can be resolved as $\mathcal{I} = \sum_{\lambda=0}^{\infty}\ket{\lambda}\bra{\lambda}$. Using the relations $\ket{\tilde{\lambda}} = \sum_{\lambda}\ket{\lambda}\braket{\lambda|\tilde{\lambda}}$, $\ket{\lambda} = a_{\lambda}^{\dagger}\ket{\Omega}$, and $\ket{\tilde{\lambda}} = a_{\tilde{\lambda}}^{\dagger} \ket{\Omega}$, the transformation law is given by:

\begin{equation}
    a_{\tilde{\lambda}}^{\dagger} = \sum_{\lambda}\braket{\lambda|\tilde{\lambda}}a_{\lambda}^{\dagger}, \quad a_{\tilde{\lambda}} = \sum_{\lambda}\braket{\tilde{\lambda}|\lambda}a_{\lambda}
\end{equation}

\subsection{Fourier transform of field operators}

The physical interpretation provided for the creation (annihilation) operators states that they can be thought of as creating (annihilating) a particle in a state $\lambda$. In particular, this can be thought of as creating (annihilating) a particle at a dimensional site $r$, or equivalently, with a momentum $k$. This distinction is important since it is subtly related to the Heisenberg Uncertainty Principle -- the first scenario implies that the position of the particle is known with a very high certainty, and therefore is delocalised in momentum space and vice-versa. The transformation from real space to $k$-space is captured via Fourier transform of the field operators.

\begin{equation}
    \hat{c}_{r}^{(\dagger)} = \frac{1}{\sqrt{N}}\sum_{k}e^{-(+)ikr} \hat{c}_{k}^{(\dagger)}, \quad \hat{c}_{k}^{(\dagger)} = \frac{1}{\sqrt{N}}\int_{r}e^{-(+)ikr} \hat{c}_{r}^{(\dagger)}
\end{equation}

If we are dealing with discrete lattice sites, the Fourier transform has to be modified accordingly

\begin{equation}
    \hat{c}_{r}^{(\dagger)} = \frac{1}{\sqrt{N}} \sum_{k}e^{-(+)ikar}c_{k}^{(\dagger)}
\end{equation}

and $k$ lies inside the first Brillouin zone, i.e. $k \in \left[-\frac{\pi}{a},\frac{\pi}{a}\right]$ and $a$ is the lattice constant.

\section{Representation of operators}

